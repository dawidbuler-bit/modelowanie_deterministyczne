\documentclass[landscape,11pt]{beamer}
\usepackage{polski}
\usepackage[utf8]{inputenc}
\usepackage{amsmath,amsfonts, amssymb, amsthm}
\usepackage{hyperref}
\usepackage{pgf,xcolor}
\usetheme{Madrid} 
\usecolortheme{rose}
\useoutertheme{default}
\useinnertheme{default}
\usefonttheme{default}
\title[Korelacja między liczbą fauli przeciwnika a liczbą 
oddanych rzutów osobistych]{Korelacja między liczbą fauli przeciwnika a liczbą 
oddanych rzutów osobistych}
\author[Dawid Buler]{Dawid Buler}
\institute[]{nr albumu 176551\\
Matematyka rok 1 semestr 1}
\date{\today}
\begin{document}
\frame[plain]{\titlepage}
\begin{frame}[plain]{Dane}
\begin{figure}
\begin{columns}
    \begin{column}{0.6\textwidth}
    \centering
        \includegraphics[width=0.6\linewidth]{dane.png}
        \caption{Źródło:\\
        \href{https://www.nba.com/team/1610612755/schedule}{https://www.nba.com/team/1610612755/schedule}}
    \end{column}
    \begin{column}{0.3\textwidth}
    \centering
        Tabelka obok przedstawia dane z ostatnich 20 spotkań drużyny NBA.
        Opisane obok dane to faule przeciwników($x_i$) oraz rzuty osobiste($y_i$). Przeanalizujmy czy istenije korelacja między liczbą fauli przeciwników a liczbą rzutów osobistych.
    \end{column}
\end{columns}
\end{figure}
\end{frame}
\begin{frame}{Wykres punktów z doświadczenia}
    \begin{figure}
    \centering
    \includegraphics[width=1\linewidth]{wyk1.png}
\end{figure}
\end{frame}
\begin{frame}{Konieczne obliczenia}
    \begin{figure}
    \centering
    \includegraphics[width=1\linewidth]{obli.png}
    \caption{Konieczne obliczenia do przeprowadzenia analizy}
\end{figure}
\end{frame}
\begin{frame}{Dalsze obliczenia}
\begin{columns}
\begin{column}{0.3\textwidth}
    \begin{figure}
    \centering
    \includegraphics[width=0.7\linewidth]{dobli.png}
    \caption{Dalsze obliczenia}
\end{figure}
\end{column}
\begin{column}{0.6\textwidth}
Współczynnik korelacji $r$ wynosi $0,70$ oznacza to silną korelacje. 
$$-S_xS_y \leq S_{xy} \leq S_xS_y$$
$$-11,96 \leq 8,40 \leq 11,96$$
Siłę korelacji możemy stwierdzić ponieważ nasze $S_{xy}$ jest blisko prawego końca przedziału a więc związek funkcyjny jest dodatni.\\
Współczynnik determinacji, czyli $R^2$ wynosi 51\%. Oznacza to słabe dopasowanie.
\end{column}
\end{columns}
\end{frame}
\begin{frame}{Predykcja dla czterech wybranych wartości cechy x}
\centering $y=0,80+1,27x$ prosta regresji\\
($x$-liczba fauli drużyny przeciwnej )
    \begin{figure}
    \centering
    \includegraphics[width=0.8\linewidth]{pro.png}
    \caption{Tabelka z prognozą}\\
\end{figure}
\end{frame}
\subsection{Wykres punktów z doświadczenia z prostą regresji}
\begin{frame}{Wykres punktów z doświadczenia z prostą regresji}
    \begin{figure}
    \centering
    \includegraphics[width=1\linewidth]{wyk2.png}
\end{figure}
\end{frame}
\subsection{Wykres rozkładu reszt}
\begin{frame}{Wykres rozkładu reszt}
    \begin{figure}
    \centering
    \includegraphics[width=1\linewidth]{wyk3.png}
\end{figure}
\end{frame}
\section{Podsumowanie}
\begin{frame}{Podsumowanie}
\centering
\begin{enumerate}
\item Korelacja badanego zjawiska jest silna oraz dodatnia. Świadczy to zatem o zależności która jest następujaca : im więcej fauli przeciwnej drużyny tym więcej oddanych rzutów osobistych. 
\item Po wyliczeniu współczynnika determinacji jesteśmy w stanie stwierdzić, iż dopasownie modelu jest słabe.
\item Odchylenie standardowe składnika resztowego wynosi $2,995$ co również oznacza o słabym dopasowaniu modelu.
\end{enumerate}
    \end{frame}
\end{document}